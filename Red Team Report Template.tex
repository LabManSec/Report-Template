%% Options for packages loaded elsewhere
\PassOptionsToPackage{unicode}{hyperref}
\PassOptionsToPackage{hyphens}{url}
\documentclass[
]{article}
\usepackage{titling}
\renewcommand\maketitlehooka{\null\mbox{}\vfill}
\renewcommand\maketitlehookd{\vfill\null}
\usepackage{lipsum}
\usepackage[a4paper, total={6in, 8in}]{geometry}
\usepackage{xcolor}
\usepackage{amsmath,amssymb}
\renewcommand{\baselinestretch}{1.5} 
\setcounter{secnumdepth}{-\maxdimen} % remove section numbering
\usepackage{iftex}
\ifPDFTeX
  \usepackage[T1]{fontenc}
  \usepackage[utf8]{inputenc}
  \usepackage{textcomp} % provide euro and other symbols
\else % if luatex or xetex
  \usepackage{unicode-math} % this also loads fontspec
  \defaultfontfeatures{Scale=MatchLowercase}
  \defaultfontfeatures[\rmfamily]{Ligatures=TeX,Scale=1}
\fi
\usepackage{lmodern}
\usepackage{beraserif}
\usepackage{berasans}
\usepackage{beramono}
\ifPDFTeX\else
  % xetex/luatex font selection
\fi
% Use upquote if available, for straight quotes in verbatim environments
\IfFileExists{upquote.sty}{\usepackage{upquote}}{}
\IfFileExists{microtype.sty}{% use microtype if available
  \usepackage[]{microtype}
  \UseMicrotypeSet[protrusion]{basicmath} % disable protrusion for tt fonts
}{}
\makeatletter
\@ifundefined{KOMAClassName}{% if non-KOMA class
  \IfFileExists{parskip.sty}{%
    \usepackage{parskip}
  }{% else
    \setlength{\parindent}{0pt}
    \setlength{\parskip}{6pt plus 2pt minus 1pt}}
}{% if KOMA class
  \KOMAoptions{parskip=half}}
\makeatother
\usepackage{graphicx}
\makeatletter
\newsavebox\pandoc@box
\newcommand*\pandocbounded[1]{% scales image to fit in text height/width
  \sbox\pandoc@box{#1}%
  \Gscale@div\@tempa{\textheight}{\dimexpr\ht\pandoc@box+\dp\pandoc@box\relax}%
  \Gscale@div\@tempb{\linewidth}{\wd\pandoc@box}%
  \ifdim\@tempb\p@<\@tempa\p@\let\@tempa\@tempb\fi% select the smaller of both
  \ifdim\@tempa\p@<\p@\scalebox{\@tempa}{\usebox\pandoc@box}%
  \else\usebox{\pandoc@box}%
  \fi%
}
% Set default figure placement to htbp
\def\fps@figure{htbp}
\makeatother
\setlength{\emergencystretch}{3em} % prevent overfull lines
\providecommand{\tightlist}{%
  \setlength{\itemsep}{0pt}\setlength{\parskip}{0pt}}
\usepackage{bookmark}
\IfFileExists{xurl.sty}{\usepackage{xurl}}{} % add URL line breaks if available
\urlstyle{same}
\hypersetup{
  pdftitle={Red Team Report},
  hidelinks,
  pdfcreator={LaTeX via pandoc}}

\title{Red Team Report - [ORGANIZATION]}
\author{[PENTEST COMPANY] \\ \small [NAME], [POSITIOn] }
\date{[DATE]}

\begin{document}
\begin{titlepage}
\maketitle
\end{titlepage}

\tableofcontents
\newpage

\subsection{Executive Summary}\label{executive-summary}

\lipsum[1-3]
\pagebreak

\subsection{Methodology and Goals}\label{methodology-and-goals}

\lipsum

\subsection{Scenario and Scope}\label{scenario-and-scope}

\subsubsection{Scenario}\label{scenario}

\lipsum[1-3]

\subsubsection{Scope}\label{scope}

\lipsum[2][2-6]

\begin{itemize}
\tightlist
\item
  \texttt{[DOMAIN]}
  \begin{itemize}
  \tightlist
  \item
    \texttt{[SERVER]}
  \end{itemize}
\item
  \texttt{[DOMAIN]}
  \begin{itemize}
  \tightlist
  \item
    \texttt{[SERVER]}
  \item
    \texttt{[SERVER]}
  \item
    \texttt{[SERVER]}
  \item
    \texttt{[SERVER]}
  \end{itemize}
\end{itemize}

\subsection{Attack Narrative}\label{attack-narrative}

The following section outlines the sequence of events that occurred over
the course of the engagement to achieve the goals defined for the
operator.

\subsubsection{\texorpdfstring{Phase 1: [TITLE]}{Phase 1: [TITLE]}}\label{phase-1}

\lipsum[1-3]

\subsubsection{\texorpdfstring{Phase 2: [TITLE]}{Phase 2: [TITLE]}}\label{phase-2}
\lipsum[4-6]

\subsubsection{\texorpdfstring{Phase 3: [TITLE]}{Phase 3: [TITLE}}\label{phase-3}

\lipsum[7-9]


\subsubsection{\texorpdfstring{Phase 4: [TITLE]}{Phase 4: [TITLE]}}\label{phase-4}
\lipsum[10-14]

\pagebreak

\subsection{Observations and Mitigation
Opportunities}\label{observations-and-mitigation-opportunities}

The following section is intended to discuss specific scenarios that
contributed to the compromise of [COMPANY]'s
Active Directory Environment. The observations include actions that
allowed the operator to conduct specific actions within the environment
or enabled the operator to conduct attacks in a stealthy, uninhibited
manner.

\subsubsection{Observation 1: [TITLE]}\label{observation-1}

\lipsum[1][1-3]

\subsubsection{Mitigation Opportunity}\label{mitigation-opportunity}

\lipsum[1][3-5]

\subsubsection{Observation 2: [TITLE]}\label{observation-2}

\lipsum[2][1-3]

\subsubsection{Mitigation Opportunity}\label{mitigation-opportunity-1}

\lipsum[2][3-10]

\subsection{Detection Opportunities and
Recommendations}\label{detection-opportunities-and-recommendations}

In certain scenarios, mitigation may not be practical or feasible for an
organization. Organizations should seek to remedy these mitigation gaps
through the development of security detections and alerts. These alerts
do not prevent malicious actions from taking place, but can be reviewed
by security analysts to identify potentially malicious activity.

\subsubsection{Detection Opportunity 1: [TITLE]}\label{detection-opportunity-1}

\lipsum[1][1-5]

\subsubsection{Detection Logic}\label{detection-logic}

\lipsum[1][5-12]
\pagebreak

\subsection{Conclusion}\label{conclusion}

\lipsum[1]

\lipsum[2]
\pagebreak

\subsection{Appendix A: Scripts and Tools
Utilized}\label{appendix-a-scripts-and-tools-utilized}

Below are a list of tools and programs utilized by the operator during
the engagement. For publicly-available tooling, hyperlinks are provided
to relevant GitHub repositories.

Built-in Tooling

\begin{itemize}
\tightlist
\item
  \texttt{[TOOL]}
\item
  \texttt{[TOOL]}
\item
  \texttt{[TOOL]}
\end{itemize}

Publicly-available Tooling

\begin{itemize}
\tightlist
\item
  \texttt{[TOOL]} (\href{http://your/link/here}{\underline{link}})
\item
  \texttt{[TOOL]} (\href{http://your/link/here}{\underline{link}})

\end{itemize}

\end{document}
